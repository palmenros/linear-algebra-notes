% !TeX root = ../algebra_lineal.tex

\documentclass[../algebra_lineal.tex]{subfiles}

\begin{document}

\section{Semejanza y equivalencia}

En el capítulo anterior hemos demostrado que a cada aplicación lineal, fijada una base, le corresponde una matriz asociada con la que podemos operar más cómodamente. Vimos también, que dicha matriz \textbf{depende de la base}. Hay algunas matrices con las que es más fácil de operar que con otras, por ejemplo, si tiene muchos ceros. 

Fijada una base, no podemos cambiar la matriz que representa una aplicación lineal, pero sí que podemos buscar una base en la que la matriz asociada a la aplicación lineal sea lo más simple posible, y trabajaremos con la matriz en dicha base. Generalmente, nos gustaría sin embargo obtener las entradas y salidas en cierta base conocida. Informalmente, el procedimiento es el siguiente: 

\begin{enumerate}
    \item \textit{Traducimos} el vector de la base de entrada $\mathcal{B}_e$ a la base en la que la matriz se puede operar de forma cómoda $\mathcal{B}_c$.
    \item \textit{Operamos} con la matriz en la base $\mathcal{B}_c$, por ejemplo, aplicando la transformación lineal al vector.
    \item \textit{Traducimos de vuelta} el vector de la base $\mathcal{B}_c$ a la base de salida $\mathcal{B}_s$.
\end{enumerate}

Estas traducciones se pueden realizar mediante matrices de cambio de base, que ya vimos que formalmente corresponden a la aplicación lineal identidad. 

Formalicemos esta introducción informal. Sean $V, V'$ dos $\K$-espacios vectoriales y sea $\mathcal{B}_V=\set{\vlst{u}{n}}$ una base de $V$ y $\mathcal{B}_{V'}=\set{\vlst{v}{m}}$ una base de $V'$. Sea $f : V \to V'$ una aplicación lineal, y consideremos su matriz $A$ asociada respecto a las bases $\mathcal{B}_V$ y $\mathcal{B}_{V'}$.

\[
    A = \mathcal{M}_{\set{\vu_i} \set{\vv_{j}}}\parens{f} \in \mathcal{M}_{m\times n}\parens{\K}  
\]

Estas bases y la matriz $A$ harán la función de las bases y matriz \textit{cómodas} en la introducción informal, las bases en las que nos gustaría trabajar con la aplicación lineal. Sin embargo, debemos aplicar la aplicación lineal $f$ a vectores entre otras bases distintas $\mathcal{B'}_{V} = \set{\vlst{x}{n}}$ de $V$ y $\mathcal{B'}_{V'} = \set{\vlst{y}{m}}$ de $V'$. Consideramos la matriz $B$ asociada a $f$ en estas bases.
\[
    B = \mathcal{M}_{\set{\vx_i} \set{\vy_{j}}}\parens{f} \in \mathcal{M}_{m\times n}\parens{\K}  
\]
Sabemos que podemos multiplicar un vector por $B$ para transformarlo por $f$ en las bases $\mathcal{B'}_{V}$ y $\mathcal{B'}_{V'}$, pero puede ser que la matriz $B$ sea mucho más complicada que la matriz $A$. Podemos considerar entonces el siguiente diagrama

%Babel conflict with tikzc
\shorthandoff{"}

\begin{equation*}
% https://tikzcd.yichuanshen.de/#N4Igdg9gJgpgziAXAbVABwnAlgFyxMJZABgBpiBdUkANwEMAbAVxiRADUB9YAHR9540YAY2BMAvpyx9x4kONLpMufIRQBGclVqMWbYOwDkkgQKGiakgFYy5CpdjwEiZddvrNWiDtz5mRwAAektI8svKKIBiOqkSabtQeet4Gxr78fObAAJ4htvLaMFAA5vBEoABmAE4QALZIZCA4EEiaOp5sFQAEfFVYxQAWOHAAjkz9dFU1AO5dAIIRlTX1iABM1M1IAMyJul4g3b39Q6PjxZMzXQBCiwfLSOtNLYiNDHQARjAMAArKTmogBgwCo4EC7DreLBQTjsHo8PqDYZjCZTCCzADCXVu1TqrQ2zx27WSICh3CM4jhCJOyPOqNmABEAHrAAC06jsFHEQA
\begin{tikzcd}
    V_{\{\vec{u}_i\}} \arrow[blue, r, "f \rightsquigarrow A"]                                       & {V'}_{\{\vec{v}_j\}} \arrow[blue, d, "id_{V'} \rightsquigarrow D^{-1}"] \\
    V_{\{\vec{x}_i\}} \arrow[r, "f \rightsquigarrow B"] \arrow[blue, u, "id_V \rightsquigarrow C "] & {V'}_{\{\vec{y}_i\}}                                             
\end{tikzcd}
\end{equation*}

\shorthandon{"}%

En el diagrama representamos en las transiciones a la izquierda de la flecha la función y a la derecha la matriz asociada a dicha aplicación lineal en las bases correspondientes. Vemos que hay dos formas de ir desde $V_{\set{\vx_i}}$ (abajo a la izquierda) a ${V'}_{\set{\vy_i}}$ (abajo a la derecha), o bien usando directamente la matriz $B$, o usando las matrices $C$ y $D^{-1}$ de cambio de base que ahora definiremos y la matriz cómoda $A$. $C$ es la matriz de base de $\mathcal{B'}_V$ a $\mathcal{B}_V$, es decir
\[
\parens{\vlst{x}{n}} = \parens{\vlst{u}{n}} \cdot C, \spc \spc \spc \underbrace{C}_{\in \glk{n}} \in \mgk{n}{n}
\]
Análogamente, $D$ es la matriz de cambio de base de  $\mathcal{B'}_{V'}$ a $\mathcal{B}_{V'}$, es decir
\[
\parens{\vlst{y}{m}} = \parens{\vlst{v}{m}} \cdot D, \spc \spc \spc \underbrace{D}_{\in \glk{m}} \in \mgk{m}{m}
\]

En vez de utilizar $D$, utilizamos el cambio de base inverso $D^{-1}$, por lo que
\[
    \parens{\vlst{v}{m}} = \parens{\vlst{v}{m}} \cdot \underbrace{ D \cdot D^{-1} }_{I_{m \times m}} = \parens{\parens{\vlst{v}{m}} \cdot D} \cdot D^{-1} = \parens{\vlst{y}{m}} D^{-1}
\]
En notación funcional, hemos demostrado que $f = id_{V'} \circ f \circ id_{V}$, lo cual es obvio y no demasiado interesante, pero a nivel matricial hemos demostrado que 
\[
    B = D^{-1} \cdot A \cdot C
\]

Esta es la idea principal con la que vamos a trabajar en el capítulo. Empezamos con una definición útil.

\begin{definition}
    Dos matrices $A, B, \in \mgk{m}{n}$ son equivalentes si $\exists C \in \glk{n}$ y $\exists C \in \glk{m}$ tales que $B=D^{-1} \cdot A \cdot C$.
\end{definition}

Veamos que ser equivalente es una relación de equivalencia.

\begin{proposition}
    Que dos matrices sean equivalentes es una relación de equivalencia.
\end{proposition}

\begin{proof}
    Estudiemos por separado las propiedades de relación de equivalencia.
    \begin{enumerate}
        \item Reflexividad: $A \sim A$ porque $A = {I_{m \times m}}^{-1} \cdot A \cdot I_{m \times m}$
        \item Simetría: Si $\exists C \in \glk{n}, D \in \glk{m}$ tal que $B=D^{-1}\cdot A \cdot C$, entonces
            \[
                D \cdot B \cdot C^{-1} = \underbrace{D \cdot D^{-1}}_{I_{m \times m}} \cdot A \cdot \underbrace{C \cdot C^{-1}}_{I_{m \times m}} = A
            \]
        \item Transitividad:
        \[
            \begin{cases}
                B \sim A \\
                A \sim G 
            \end{cases}  \Rightarrow
            \begin{cases}
                B = D^{-1} \cdot A \cdot C \\
                A = E^{-1} \cdot G \cdot F
            \end{cases}
        \]
        Entonces,
        \[
            B = D^{-1} \cdot A \cdot C = \underbrace{\parens{D^{-1} E^{-1}}}_{\in \glk{m}} \cdot G \cdot \underbrace{ \parens{F \cdot C} }_{\in \glk{n}} 
        \]
    \end{enumerate}
\end{proof}
La siguiente proposición nos da una forma muy fácil de calcular una \textit{forma canónica} para la equivalencia. Es decir, una matriz equivalente muy sencilla con la que poder comparar si dos matrices cualesquiera son equivalentes o no.

\begin{proposition}
    \label{forma_canonica_equivalencia}
    Sea $f : V \to V'$ una aplicación lineal con $\ran{f} = r$. Entonces, existen bases $\mathcal{B}$ y $\mathcal{B'}$ en $V$ y $V'$ tal que 
    \[
        \mathcal{M}_{\mathcal{B}, \mathcal{B'}}\parens{f} = 
        \begin{pmatrix}
            I_{r \times r}
            & \rvline & 0_{r \times \parens{n-r}} \\
          \hline
          0_{\parens{m-r} \times r} & \rvline &
          0_{\parens{m-r} \times \parens{n-r}}
          \end{pmatrix}
    \]

\end{proposition}

\begin{proof}
    Sea $n=\dim{V}$ y sea $\set{\vx_{r+1}, \dots, \vx_{n}}$ una base de $\ker{f}$, que extendemos hasta una base $\mathcal{B}=\set{\cvlst{x}{n}{r}}$ de $V$. Entonces, el conjunto 
    \[
       \{ f\parens{\vx_1}, f\parens{\vx_2}, \dots, f\parens{\vx_r}, \underbrace{\cancel{f\parens{\vx_{r+1}}}}_{\zv}, \underbrace{\cancel{f\parens{\vx_{r+2}}}}_{\zv}, \dots, \underbrace{\cancel{f\parens{\vx_{n}}}}_{\zv} \}  
    \]
    es sistema de generadores de $\im{f}$, por tanto como $\dim{\im{f}} = r$, $\set{f\parens{\vx_1}, f\parens{\vx_2}, \dots, f\parens{\vx_r}}$ es base de $\im{f}$, y la podemos extender a una base $\mathcal{B'} = \set{f\parens{\vx_1}, f\parens{\vx_2}, \dots, f\parens{\vx_r}, \vy_{r+1}, \dots, \vy_n}$ de $V'$. Entonces, la matriz $\mathcal{M}_{\mathcal{B}, \mathcal{B'}}\parens{f}$ cumple el enunciado, porque si $i \le r$, la imagen del $i$-ésimo vector $\vx_i$ de la base $\mathcal{B}$ es el $i$-ésimo vector $f\parens{\vx_i}$ de la base $\mathcal{B}'$. Si $i > r$, entonces $\vx_i \in \ker{f}$, por lo que la imagen es el vector nulo.  
\end{proof}

\begin{corollary}
    Dos matrices $A$ y $B$ son equivalentes si y solo si $\ran{A} = \ran{B}$.
\end{corollary}
\begin{proof}
    Basta obtener la forma canónica de la proposición \ref{forma_canonica_equivalencia} y compararlas entre ellas.
\end{proof}


\end{document}
% !TeX root = ../algebra_lineal.tex

\documentclass[../ecuaciones_diferenciales.tex]{subfiles}

\begin{document}

\section{Notación}

¿Explicar aquí notación del tal que? ¿Explicar aquí la notación del subset (significa subseteq realmente)?

\section{Cuerpos}

Sea $\K$ un cuerpo abeliano de característica distinta de 2.

$\dlstp{\vx}{p}{i} = \lspan{H} \neq \emptyset$


$\exists a \in \K \st$

\begin{definition}
    Decimos que $\K$ es un cuerpo abeliano si cumple las siguientes propiedades en $\set{u : v \in V}$

    \begin{enumerate}
        \item $0 \in \K$ 
    \end{enumerate}

    \begin{theorem}[Teorema de Cauchy]
        Teorema de Cauchy.        
    \end{theorem}
    \begin{proof}
        Esto es la demostración
    \end{proof}

\end{definition}

\end{document}

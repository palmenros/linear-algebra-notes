% !TeX root = ../algebra_lineal.tex

\documentclass[../ecuaciones_diferenciales.tex]{subfiles}

\begin{document}

\section{Notación}

¿Explicar aquí notación del tal que? ¿Explicar aquí la notación del subset (significa subseteq realmente)? ¿Explicar aquí que los superíndices no son potencias?

\section{Cuerpos}

Para formalizar el concepto de espacio vectorial, deberemos antes formalizar el conjunto de escalares, es decir, un conjunto ``de dimensión uno'', signifique lo que signifique eso, que podremos utilizar para multiplicar (``escalar'') vectores. Habitualmente usaremos $\R$ o $\Complex$, pero en esta sección vamos a intentar extraer las propiedades relevantes de estos dos conjuntos para definir una estructura algebraica abstracta, el \textit{cuerpo}, del que únicamente conoceremos sus propiedades.

Informalmente, las propiedades más relevantes son la existencia de las operaciones suma y multiplicación, que estas sean algebraicamente cerradas
(es decir, que al sumar o multiplicar dos elementos del cuerpo obtengamos otro elemento del cuerpo), que cumplan las propiedades asociativa, conmutativa
\footnote{En ocasiones se distinguen a los cuerpos que cumplen la propiedad conmutativa como \textit{cuerpos abelianos}, en este texto todos los cuerpos
 serán conmutativos} y distributiva, que existan elementos neutros (0 para la suma y 1 para la multiplicación), que exista el concepto de \textit{división} (es decir, que cada elemento tenga un inverso multiplicativo), el concepto de \textit{resta} (es decir, que cada elemento tenga un opuesto aditivo).

Analicemos ahora la definición formal.

\begin{definition}[Cuerpo]
    Decimos que $\K$ es un cuerpo si existen dos aplicaciones $+:\K \times \K \to \K$, $\cdot:\K \times \K \to \K$ llamadas respectivamente suma y multiplicación que satisfacen las siguientes propiedades:
    \begin{enumerate}
        \item Asociatividad de la suma y multiplicación: $a+(b+c)=(a+b)+c$, $a\cdot(b\cdot c)=(a\cdot b)\cdot c$
        \item Conmutatividad de la suma y multiplicación: $a+b=b+a$, $a\cdot b = b \cdot a$
        \item Distributividad de la multiplicación sobre la suma: $a \cdot (b+c) = (a \cdot b) + (a \cdot c)$
        \item Elemento neutro para la suma: $\exists 0 \in \K \st \forall a \in \K, \spc a+0=a$
        \item Elemento identidad para la multiplicación:  $\exists 1 \in \K \st \forall a \in \K, \spc a\cdot 1=a$
        \item \label{cuerpo_opuesto} Opuesto aditivo: $\forall a \in \K, \spc \exists {(-a)} \in \K \st a + (-a) = 0 $.
        \item \label{cuerpo_inverso} Inverso multiplicativo: $\forall a \neq 0 \in \K, \spc \exists {(a^{-1})} \in \K \st a \cdot (a^{-1}) = 1$
    \end{enumerate}    
\end{definition}

\begin{notation}
    En las propiedades \ref{cuerpo_opuesto} y \ref{cuerpo_inverso} de los opuestos e inversos, hemos realizado un pequeño abuso de notación por motivos de claridad. Lo que realmente indica la propiedad es que para cualquier $a \in \K$ existe otro elemento $b_a \in \K$ (indicamos con el subíndice que depende de $a$) tal que $a + b = 0$. Denotamos al elemento $b_a$ como $-a$. Análogamente hacemos lo propio con el inverso multiplicativo. 
\end{notation}

\begin{remark}
    Sea cual sea el cuerpo, contiene a todos los números naturales ya que si $n \in \N \implies n = 1 + 1 + \dots + 1 \in \K$ porque cada uno de los $1$ pertenece a $\K$, por lo que la suma pertenece también.
\end{remark}

Con esta definición de cuerpo hemos ampliado significativamente el rango de conjuntos que podemos usar. Sin embargo, surgen algunos casos patológicos, como por ejemplo $\Z_2 = \set{0, 1}$, el cuerpo de los enteros módulo 2 (es decir, se ejecuta la aritmética usualmente y finalmente colapsamos los pares al 0 y los impares al 1). Se deja como ejercicio al lector demostrar que efectivamente $\Z_2$ es un cuerpo (de hecho, $\Z_p$ es un cuerpo para todo $p$ primo, algo fuera del alcance de este texto). La patología de $\Z_2$ radica en que $1+1=0$. A menudo querremos dividir entre 2 en variadas demostraciones, por lo que impondremos un requisito especial a los cuerpos que utilizaremos. Para ello, debemos introducir la definición de \textit{característica}.

\begin{definition}
    Llamamos característica de $\K$ al mínimo número de veces que es necesario sumar el elemento identidad 1 en una suma para obtener el elemento neutro 0. Es decir, la característica es el entero positivo $n$ más pequeño tal que:
    \[
        \underbrace{1+\dots+1}_{n\, \, \mathrm{sumandos}} = 0
    \]  
\end{definition}

A partir de ahora, salvo que mencionemos expresamente lo contrario, siempre hablaremos de cuerpos $\K$ de característica distinta de 2 (es decir, $1+1 \ne 0$).

\section{Espacios vectoriales}

Al igual que hicimos con los cuerpos, vamos a definir los espacios vectoriales usando únicamente las propiedades que lo definen. Más adelante relacionaremos esta definición con el concepto intuitivo de espacio vectorial utilizando coordenadas.

Al contrario que hicimos con los cuerpos, para definir un espacio vectorial necesitaremos un cuerpo $\K$ de característica distinta de 2, que utilizaremos para multiplicar los vectores, por lo que el cuerpo aparecerá en la definición de espacio vectorial.

\begin{definition}[Espacio vectorial]
    \label{vector_space_definition}
    Un conjunto $V \neq \emptyset$ es un espacio vectorial sobre $\K$ (también diremos que es un $\K$-espacio vectorial) si existen las siguientes operaciones en $V$ (suma y multiplicación por escalar) 
    
    \begin{multicols}{2}
        \noindent
        \begin{align*} +: V \times V &\to V \\
                (\vx, \vy) &\mapsto \vx + \vy
        \end{align*}
        \begin{align*}
            \cdot: \K \times V &\to V \\
                (a, \vx) &\mapsto a \vx
        \end{align*}
    \end{multicols}

    y satisfacen las siguientes propiedades:

    \begin{enumerate}
        \item Conmutatividad de la suma: $\forall \vx, \vy \in V, \spc \vx + \vy = \vy + \vx$
        \item Asociatividad de la suma: $\forall \vx, \vy, \vz \in V, \spc \vx + (\vy + \vz) = (\vx + \vy) + \vz$
        \item Elemento neutro: $\exists \zv \in V \st \forall \vx \in V, \spc \vx + \zv = \vx$
        \item Elemento opuesto: $\forall \vx \in V, \spc \exists {({-x})} \in V \st  \vx + (-\vx) = \zv$
        \item Distributiva vectorial: $\forall a \in \K, \spc \forall \vx, \vy \in V, \spc a(\vx + \vy) = a\vx + a \vy$
        \item Distributiva escalar: $\forall a, b\in \K, \spc \forall \vx \in V, \spc (a+b)\vx = a\vx + b\vx$ 
        \item Asociativa: $\forall a, b \in \K, \spc \forall \vx \in V, \spc a \cdot (b \cdot \vx) = (a\cdot b) \cdot \vx $
        \item \label{espacio_vectorial_unidad} Elemento unidad: $\forall \vx \in V, \spc 1 \cdot \vx = \vx$ 
    \end{enumerate}

\end{definition}

\begin{remark}
    Es importante distinguir el elemento neutro $0 \in \K$ del cuerpo escalar, del elemento neutro $\zv \in V$ del espacio vectorial, ya que uno es un vector y otro un escalar. Sin embargo, el elemento identidad $1 \in \K$ de la propiedad \ref{espacio_vectorial_unidad} sí que es un escalar (se suman vectores, y se multiplican vectores con escalares).  
\end{remark}

\begin{notation}
    Si $V$ es un \kvspace, llamaremos a los elementos de V \textit{vectores} y a los elementos de $\K$ escalares.
\end{notation}

A partir de ahora, si no indicamos otra cosa, $V$ será un \kvspace, siendo $\K$ un cuerpo de característica distinta de 2.

Es posible que esta definición abstracta de espacio vectorial parezca al lector algo árida, sin embargo es una herramienta muy potente que nos permitirá desarrollar una teoría que se puede aplicar a cualquier conjunto que satisfaga la definición de espacio vectorial, no únicamente a $\K^n$. Ejemplos de espacios vectoriales muy útiles en las matemáticas son los polinomios $\K[x]$ \footnote{Denotaremos como $\K[x]$ el conjunto de polinomios de cualquier grado sobre la variable $x$, es decir, polinomios de la forma, $a_n x^n + \dots + a_1 x + a_0$ con $a_i \in \K \spc \forall i \in {1, \dots, n}$}, las funciones continuas, las soluciones de ecuaciones diferenciales lineales, etc. Si desarrollamos todo desde las propiedades de espacio vectorial, podremos aplicar toda la teoría que desarrollemos a todos esos espacios vectoriales y muchos otros, sin tener que distinguir casos durante el desarrollo de la teoría.

A pesar de la potencia de esta aproximación a la definición de espacio vectorial, debemos demostrar incluso las cosas más nimias desde las propiedades. Comenzamos demostrando algunas propiedades inmediatas.

\begin{proposition}
    Sea $V$ un \kvspace. Entonces, las siguientes propiedades son ciertas:
    \begin{enumerate}
        \item $a \cdot \zv = \zv \spc \forall a \in \K$
        \item $a(-\vx) = -a\cdot \vx \spc \forall a \in \K$
        \item $0 \cdot \vx = \zv \spc \forall \vx \in V$
        \item \label{propiedad_cuatro_espacios_vectoriales} $(-a)\cdot \vx = -a\vx$
        \item Si $\vx \ne 0$, entonces $a \cdot \vx = \zv \implies a = 0$
    \end{enumerate}
\end{proposition}

\begin{proof}
    \begin{enumerate}[wide, labelwidth=0pt, labelindent=0pt]
        \item Primero observamos que $a\cdot \zv = a(\zv + \zv) = a \cdot \zv + a \cdot \zv $. Restando a ambos lados $a\cdot \zv$ obtenemos que 
            \[\zv = a\cdot \zv + (-a\cdot \zv) = (a \cdot \zv + a \cdot \zv) + (-a \cdot \zv) =  a \cdot \zv + \underbrace{(a \cdot \zv + (-a \cdot \zv))}_{\zv} = a \cdot \zv \]
        \item Primero nos percatamos de que $\zv = a \cdot \zv = a(\vx + (-\vx)) = a\vx +a\cdot(-\vx)$. Como $\zv$ es igual a dicha expresión, podemos sustituir la expresión de la derecha en cualquier lugar en el que aparezca $\zv$ por lo que
         \[ -a\vx = -a\vx + \zv = -a\vx + (a\vx + a(-\vx)) = \underbrace{(-a\vx + a\vx)}_{\zv} + a(-\vx) = a(-\vx) \] 
        \item En primer lugar vemos que $0 \cdot \vx = (0+0)\vx = 0 \cdot \vx + 0 \cdot \vx$. De nuevo, podremos sustituir eso en la siguiente ecuación
        \[\zv  = 0 \cdot \vx +(-0\cdot \vx) = (0\cdot \vx + 0\cdot \vx) + (-0\cdot \vx) =  0\cdot \vx + \underbrace{(0\cdot \vx + (-0\cdot \vx))}_{\zv} = 0\cdot\vx\]
        \item Observamos que $\zv = 0 \cdot \vx = (a + (-a))\vx = a\vx + (-a)\vx$. Sumar $\zv$ es equivalente a no hacer nada, por lo que podremos sumar la expresión de la derecha sin cambiar ningún vector
        \[-a\vx = -a\vx + (a\vx + (-a)\vx) = \underbrace{(-a\vx + a\vx)}_{\zv} + (-a)\vx = (-a)\vx \]
        \item Lo demostraremos por reducción al absurdo. Supongamos que $a \neq 0$, entonces $\frac{1}{a} \in \K$. Por tanto
        \[\vx = \underbrace{\left(\frac{a}{a} \right)}_{1} \vx = \frac{1}{a} (a \cdot \vx) = \frac{1}{a} \cdot \zv = \zv\]
        Hemos llegado a que $\vx = 0$, lo que es una contradicción.
    \end{enumerate}
\end{proof}

\begin{notation}
    Utilizaremos la notación usual en la que restar significa sumar el opuesto, es decir, $\vx - \vy = \vx + (-\vy)$
\end{notation}

\begin{remark}
    Si particularizamos la propiedad \ref{propiedad_cuatro_espacios_vectoriales} para $a=1$, obtenemos que $-\vx = (-1)\vx$, por lo que para obtener el vector opuesto basta con multiplicar por el opuesto del escalar $1$ (es decir, por $-1$).
\end{remark}

\section{Subespacios vectoriales}

\end{document}

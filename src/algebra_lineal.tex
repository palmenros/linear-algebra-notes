\documentclass[a4paper, 10pt, openany]{report}

\usepackage{geometry}
\usepackage[spanish]{babel}
\usepackage{parskip}
\usepackage{delarray}
\usepackage{faktor}
\usepackage{graphicx}
\usepackage{cancel}
\usepackage{systeme}
\usepackage{tikz-cd}

\graphicspath{ {./teoria/img} }

% básicos
\usepackage{subfiles} 
\usepackage{hyperref} % último en cargar

% personalizados
\usepackage{general}
\usepackage{conjuntos}
\usepackage{teoremas}
\usepackage{calculo}
\usepackage{algebra}
\usepackage{custom_definitions}

\title{Álgebra lineal}
\author{Pedro Palacios Almendros}
\date{Curso 2018--2019}

\begin{document}

\maketitle

\tableofcontents

\chapter{Espacios vectoriales}
\subfile{teoria/espacios_vectoriales}
\chapter{Aplicaciones lineales}
\subfile{teoria/aplicaciones_lineales}
\chapter{Matrices. Determinantes. Sistemas de ecuaciones lineales.}


%\setcounter{chapter}{3}

\chapter{Clasificación de endomorfismos. Forma canónica de Jordan.}
\subfile{teoria/clasificacion_endomorfismos}

\end{document}
